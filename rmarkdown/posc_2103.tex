\documentclass[11pt,]{article}
\usepackage[margin=1in]{geometry}
\newcommand*{\authorfont}{\fontfamily{phv}\selectfont}
\usepackage{lmodern}
\usepackage{abstract}
\renewcommand{\abstractname}{}    % clear the title
\renewcommand{\absnamepos}{empty} % originally center
\newcommand{\blankline}{\quad\pagebreak[2]}

\usepackage{enumitem, xcolor}
\definecolor{astate}{HTML}{cc092f}

\usepackage{sectsty} 
\usepackage[normalem]{ulem} 
\sectionfont{\Large\bfseries\sffamily\color{astate}} 
\subsectionfont{\large\bfseries\sffamily\color{astate}}
\subsubsectionfont{\normalsize\bfseries\sffamily\color{astate}}



\providecommand{\tightlist}{%
  \setlength{\itemsep}{0pt}\setlength{\parskip}{0pt}} 
\usepackage{longtable,booktabs}

\newcommand{\tab}{\hspace*{2.8em}} %allows for custom spacing using \tab

\usepackage{parskip}
\usepackage{titlesec}
\titlespacing\section{0pt}{12pt plus 4pt minus 2pt}{6pt plus 2pt minus 2pt}
\titlespacing\subsection{0pt}{12pt plus 4pt minus 2pt}{6pt plus 2pt minus 2pt}

\titleformat*{\subsubsection}{\normalsize\itshape}

\usepackage{titling}
\setlength{\droptitle}{-.25cm}

%\setlength{\parindent}{0pt}
%\setlength{\parskip}{6pt plus 2pt minus 1pt}
%\setlength{\emergencystretch}{3em}  % prevent overfull lines 

\usepackage[T1]{fontenc}
\usepackage[utf8]{inputenc}
\usepackage[super]{nth}

% ---- FONTS
\usepackage{fontspec} 
\usepackage{xunicode}
\usepackage{xltxtra}
\defaultfontfeatures{Mapping=tex-text} % converts LaTeX specials (``quotes'' --- dashes etc.) to unicode
\setromanfont[Mapping={tex-text}, 
	Numbers={OldStyle},
	BoldFont=Georgia Bold,
	Ligatures={Common}]{Georgia}
\setsansfont[Mapping=tex-text,
	Ligatures={Common}, 
	Colour=cc092f]{Tahoma}
\setmonofont[Mapping=tex-text,Scale=0.72]{Menlo} 
\usepackage{fontawesome}

\usepackage{fancyhdr}
\pagestyle{fancy}
\usepackage{lastpage}
\renewcommand{\headrulewidth}{0.3pt}
\renewcommand{\footrulewidth}{0.0pt} 
\lhead{}
\chead{}
\rhead{\footnotesize POSC 2103--10A: Introduction to United States
Government -- Fall 2023}
\lfoot{}
\cfoot{\small \thepage/\pageref*{LastPage}}
\rfoot{}

\fancypagestyle{firststyle}
{
\renewcommand{\headrulewidth}{0pt}%
   \fancyhf{}
   \fancyfoot[C]{\small \thepage/\pageref*{LastPage}}
}

%\def\labelitemi{--}
%\usepackage{enumitem}
%\setitemize[0]{leftmargin=25pt}
%\setenumerate[0]{leftmargin=25pt}

\newcommand{\HRule}{\rule{\linewidth}{0.5mm}} 

% for the tables
\usepackage{booktabs}
\usepackage{longtable}
\usepackage{array}
\usepackage{multirow}
\usepackage{wrapfig}
\usepackage{float}
\usepackage{colortbl}
\usepackage{pdflscape}
\usepackage{tabu}
\usepackage{threeparttable}
\usepackage{threeparttablex}
\usepackage[normalem]{ulem}
\usepackage{makecell}
\usepackage{xcolor}


\makeatletter
\@ifpackageloaded{hyperref}{}{%
\ifxetex
  \usepackage[setpagesize=false, % page size defined by xetex
              unicode=false, % unicode breaks when used with xetex
              xetex]{hyperref}
\else
  \usepackage[unicode=true]{hyperref}
\fi
}
\@ifpackageloaded{color}{
    \PassOptionsToPackage{usenames,dvipsnames}{color}
}{%
    \usepackage[usenames,dvipsnames]{color}
}
\makeatother
\hypersetup{breaklinks=true,
            bookmarks=true,
            pdfauthor={ ()},
             pdfkeywords = {},  
            pdftitle={POSC 2103--10A: Introduction to United States
Government},
            colorlinks=true,
            citecolor=blue,
            urlcolor=blue,
            linkcolor=magenta,
            pdfborder={0 0 0}}
\urlstyle{same}  % don't use monospace font for urls


\setcounter{secnumdepth}{0}





\usepackage{setspace}

\title{\LARGE\bfseries\sffamily\color{astate} POSC 2103--10A:
Introduction to United States Government}
\author{Dr.~Cameron Wimpy}
\date{Fall 2023}


\begin{document}  
	% \vspace*{1\baselineskip}
	% \hspace*{-0.1\textwidth}{\centering\includegraphics[scale=.5]{astate.png}}
	
		\maketitle
		
	
		\thispagestyle{firststyle}

%	\thispagestyle{empty}


	\noindent \begin{tabular*}{\textwidth}{ @{\extracolsep{\fill}} lr @{\extracolsep{\fill}}}


\faEnvelopeO \hspace{.2em}\texttt{\href{mailto:cwimpy@astate.edu}{\nolinkurl{cwimpy@astate.edu}}}     & \faGlobe \hspace{.2em}\href{http://cwimpy.com}{\tt cwimpy.com}\\
\faHourglass \hspace{.2em}By Zoom as
needed  & \faClockO \hspace{.2em}Online\\
\faHome \hspace{.2em}HSS:
3007C                 & \faLocationArrow \hspace{.2em}Online\\
\faUserPlus \hspace{.2em}TBD                     & \faSlack \hspace{.2em}TBD\\
	&  \\
	\hline
	\end{tabular*}
\vspace{2mm}
	


\hypertarget{course-description}{%
\section{Course Description}\label{course-description}}

This course serves as an introduction to: \textbf{The constitution,
government, and politics of the United States.} The main goal of this
course is to help you understand the structure and processes of the
American political system. This is accomplished by reading about and
analyzing:

\begin{itemize}
\tightlist
\item
  the key structures and institutions of government;
\item
  relevant political actors such as political parties, interest groups
  and the media;
\item
  the electoral process;
\item
  voters and voting behavior.
\end{itemize}

\hypertarget{general-education-goal}{%
\section{General Education Goal}\label{general-education-goal}}

\begin{itemize}
\tightlist
\item
  \textbf{Developing a strong foundation in the social sciences}:
  Students should be aware of the diverse systems developed by humans to
  manage and structure our relationships with one another. Students
  should prepare for the full range of public and private roles they are
  expected to fulfill as citizens, decision-makers and human beings in a
  democratic America and in a global society.
\end{itemize}

\hypertarget{student-learning-outcomes}{%
\section{Student Learning Outcomes}\label{student-learning-outcomes}}

\begin{enumerate}
\def\labelenumi{\arabic{enumi}.}
\item
  Explain the processes and effects of individual and group behavior
\item
  Analyze events in terms of the concepts and relational propositions
  generated by the social science tradition
\item
  Have a working knowledge of the government of the United States
\item
  Understand the challenges of governing in the United States
\item
  Explain how individuals and groups interact to produce their
  collective experience
\item
  Analyze empirical evidence to make reasoned arguments about government
  of the United States
\end{enumerate}

\clearpage

\hypertarget{course-requirements}{%
\section{Course Requirements}\label{course-requirements}}

\hypertarget{textbook}{%
\subsection{Textbook}\label{textbook}}

This textbook is required. You can search for it online, in the
bookstore, and view the publisher's website via this ISBN:
\href{http://global.oup.com/ushe/product/american-government-9780190299903}{978-0190299903}.
The book includes a student companion website that has additional
materials to help you study and further explore government in the United
States.

\hangindent=0.7cm

Krutz, Glen (2021). \emph{American Government}. Houston, TX:
OpenStax---Rice University.

\hypertarget{technology}{%
\subsection{Technology}\label{technology}}

You will need to use your A-State email and have access to Canvas If
there is any issue with these items you should me know by the second day
of class.

\hypertarget{exams}{%
\subsection{Exams}\label{exams}}

There will be two multiple-choice exams. The exams will take place via
Canvas (make sure you can access Canvas by the second day of class).
Each exam will consist of 50 questions taken from the lectures and
readings. The final exam is \underline{not} comprehensive.

\hypertarget{course-policies}{%
\section{Course Policies}\label{course-policies}}

These course policies are subject to change, but assume they are in
effect unless you hear otherwise from Dr.~Wimpy.

\hypertarget{academic-freedom}{%
\subsection{Academic Freedom}\label{academic-freedom}}

We are living in a particularly polarized period of American politics,
and that means that some of us will often disagree with differing (but
typically reasonable) points of view. Disagreement and debate is a
healthy part of our political system so please be respectful of your
classmates as they share their perspectives. Please know you have
complete academic freedom to express your opinions and thoughts on U.S.
Government. Having said that, any type of hateful or insulting
discussion will not be tolerated.

\hypertarget{grading-policy}\\
Exam \#1 & 25\%\\
\cellcolor{gray!6}{Exam \#2} & \cellcolor{gray!6}{25\%}\\
Quizzes & 25\%\\
\bottomrule
\end{tabular}
\end{table}

The grading system will follow the
\href{https://www.astate.edu/college/graduate-school/academic-policies/}{A-State
convention}. Letter grades will be assigned as follows:

\begin{table}[H]
\centering
\begin{tabular}{ll}
\toprule
Letter Grade & Range\\
\midrule
\cellcolor{gray!6}{A:} & \cellcolor{gray!6}{89.5–100}\\
B: & 79.5–89.4\\
\cellcolor{gray!6}{C:} & \cellcolor{gray!6}{69.5–79.4}\\
D: & 59.5–69.4\\
\cellcolor{gray!6}{F:} & \cellcolor{gray!6}{00.0-59.4}\\
\bottomrule
\end{tabular}
\end{table}

\hypertarget{make-up-exam-policy}{%
\subsection{Make-Up Exam Policy}\label{make-up-exam-policy}}

There are no make-ups for missed exams unless there is a valid excuse.
Don't bother asking.

\hypertarget{extra-creditgrade-change-policy}{%
\subsection{Extra Credit/Grade Change
Policy}\label{extra-creditgrade-change-policy}}

I do not give extra credit and grades will only be changed in the event
of a calculation error. Do not bother asking about anything else.

\hypertarget{attendance}{%
\subsection{Attendance}\label{attendance}}

\hypertarget{laptops-phones-tablets}{%
\subsection{Laptops, Phones, \& Tablets}\label{laptops-phones-tablets}}

Laptops and tablets are allowed in the classroom insofar as they are
used to take notes and/or accessing the online text (should you opt for
the online edition). Phones are not allowed to be out during class time
under any circumstances. If you have an emergency, please leave the
class and take care of it. Otherwise it can wait. If I see abuse of the
laptop/tablet policy I reserve the right to disallow these during class
time unless there is a documented and approved exception.

\hypertarget{academic-dishonesty-policy}{%
\subsection{Academic Dishonesty
Policy}\label{academic-dishonesty-policy}}

\hypertarget{access-and-accomodations-policy}{%
\subsection{Access and Accomodations
Policy}\label{access-and-accomodations-policy}}

\hypertarget{copyright-fair-use-statement}{%
\subsection{Copyright \& Fair Use
Statement}\label{copyright-fair-use-statement}}

The materials used in this course are subject to copyright laws. For
those materials which neither I nor ASU owns the copyright, I have
either obtained a license for use or am using the materials in a manner
that I reasonably believe is in compliance with the Fair Use exception
to the copyright laws. The other materials used in this course are
copyrighted. By this, I mean all materials generated for this class,
which include but are not limited to syllabi, exams, lectures, quizzes,
assignments, and any other document I post in Blackboard. Because these
items are copyrighted, you do not have the right to copy and distribute
any course materials unless I, or the document publishers, expressly
grant permission.

\newpage

\hypertarget{tentative-class-schedule}{%
\section{Tentative Class Schedule}\label{tentative-class-schedule}}

Important: class readings are subject to change, contingent on
mitigating circumstances and the progress we make as a class. Students
are encouraged to check the Blackboard announcements section and any
e-mails from me for updates.

\hypertarget{week-01-0821-0825-course-introduction-topic-introduction-u.s.-government}{%
\subsection{Week 01, 08/21 -- 08/25: Course Introduction, Topic
Introduction (U.S.
Government)}\label{week-01-0821-0825-course-introduction-topic-introduction-u.s.-government}}

\hypertarget{read}{%
\subsubsection{Read:}\label{read}}

\begin{itemize}
\item
  The Course Syllabus
\item
  Chapter 1
\end{itemize}

\hypertarget{week-02-0828-0901-the-constitution}{%
\subsection{Week 02, 08/28 -- 09/01: The
Constitution}\label{week-02-0828-0901-the-constitution}}

\hypertarget{read-1}{%
\subsubsection{Read:}\label{read-1}}

\begin{itemize}
\tightlist
\item
  Chapter 2
\end{itemize}

\hypertarget{week-03-0904-0908-federalism}{%
\subsection{Week 03, 09/04 -- 09/08:
Federalism}\label{week-03-0904-0908-federalism}}

\hypertarget{read-2}{%
\subsubsection{Read:}\label{read-2}}

\begin{itemize}
\tightlist
\item
  Chapter 3
\end{itemize}

\hypertarget{week-04-0911-0915-civil-liberties}{%
\subsection{Week 04, 09/11 -- 09/15: Civil
Liberties}\label{week-04-0911-0915-civil-liberties}}

\hypertarget{read-3}{%
\subsubsection{Read:}\label{read-3}}

\begin{itemize}
\tightlist
\item
  Chapter 4
\end{itemize}





\vfill{}
\vfill{}
\HRule
%%%%%%%%%%%%%%%%%%%%%%%%%%%%%%%%%%%%%%%%%%%%%%%%%%%%%%%%%%%%%%%%%%%%%%%%%%%%%%%%%%%%%%%%%%%%%%%%%%%%%%%%%%%%%%%%%%%%%%%%%%%%%%%%%%%%%%%%%%%
\begin{center}
	{\scriptsize  Last updated: \today\- •\- \href{https://cwimpy.com}{cwimpy.com}}
 \end{center}
\end{document}

\makeatletter
\def\@maketitle{%
  \newpage
%  \null
%  \vskip 2em%
%  \begin{center}%
  \let \footnote \thanks
    {\fontsize{18}{20}\selectfont\raggedright  \setlength{\parindent}{0pt} \@title \par}%
}
%\fi
\makeatother
