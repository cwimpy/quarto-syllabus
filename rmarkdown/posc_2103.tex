\documentclass[11pt,]{article}
\usepackage[margin=1in]{geometry}
\newcommand*{\authorfont}{\fontfamily{phv}\selectfont}
\usepackage{lmodern}
\usepackage{abstract}
\renewcommand{\abstractname}{}    % clear the title
\renewcommand{\absnamepos}{empty} % originally center
\newcommand{\blankline}{\quad\pagebreak[2]}

\usepackage{enumitem, xcolor}
\definecolor{astate}{HTML}{cc092f}

\usepackage{sectsty} 
\usepackage[normalem]{ulem} 
\sectionfont{\Large\bfseries\sffamily\color{astate}} 
\subsectionfont{\large\bfseries\sffamily\color{astate}}
\subsubsectionfont{\normalsize\bfseries\sffamily\color{astate}}



\providecommand{\tightlist}{%
  \setlength{\itemsep}{0pt}\setlength{\parskip}{0pt}} 
\usepackage{longtable,booktabs}

\newcommand{\tab}{\hspace*{2.8em}} %allows for custom spacing using \tab

\usepackage{parskip}
\usepackage{titlesec}
\titlespacing\section{0pt}{12pt plus 4pt minus 2pt}{6pt plus 2pt minus 2pt}
\titlespacing\subsection{0pt}{12pt plus 4pt minus 2pt}{6pt plus 2pt minus 2pt}

\titleformat*{\subsubsection}{\normalsize\itshape}

\usepackage{titling}
\setlength{\droptitle}{-.25cm}

%\setlength{\parindent}{0pt}
%\setlength{\parskip}{6pt plus 2pt minus 1pt}
%\setlength{\emergencystretch}{3em}  % prevent overfull lines 

\usepackage[T1]{fontenc}
\usepackage[utf8]{inputenc}
\usepackage[super]{nth}

% ---- FONTS
\usepackage{fontspec} 
\usepackage{xunicode}
\usepackage{xltxtra}
\defaultfontfeatures{Mapping=tex-text} % converts LaTeX specials (``quotes'' --- dashes etc.) to unicode
\setromanfont[Mapping={tex-text}, 
	Numbers={OldStyle},
	BoldFont=Minion Pro Bold,
	Ligatures={Common}]{Minion Pro}
\setsansfont[Mapping=tex-text,
	Ligatures={Common}, 
	Colour=cc092f]{Myriad Pro}
\setmonofont[Mapping=tex-text,Scale=0.72]{PragmataPro} 
\usepackage{fontawesome}

\usepackage{fancyhdr}
\pagestyle{fancy}
\usepackage{lastpage}
\renewcommand{\headrulewidth}{0.3pt}
\renewcommand{\footrulewidth}{0.0pt} 
\lhead{}
\chead{}
\rhead{\footnotesize POSC 2103--007: Introduction to United States
Government -- Spring 2020}
\lfoot{}
\cfoot{\small \thepage/\pageref*{LastPage}}
\rfoot{}

\fancypagestyle{firststyle}
{
\renewcommand{\headrulewidth}{0pt}%
   \fancyhf{}
   \fancyfoot[C]{\small \thepage/\pageref*{LastPage}}
}

%\def\labelitemi{--}
%\usepackage{enumitem}
%\setitemize[0]{leftmargin=25pt}
%\setenumerate[0]{leftmargin=25pt}

\newcommand{\HRule}{\rule{\linewidth}{0.5mm}} 

% for the tables
\usepackage{booktabs}
\usepackage{longtable}
\usepackage{array}
\usepackage{multirow}
\usepackage{wrapfig}
\usepackage{float}
\usepackage{colortbl}
\usepackage{pdflscape}
\usepackage{tabu}
\usepackage{threeparttable}
\usepackage{threeparttablex}
\usepackage[normalem]{ulem}
\usepackage{makecell}
\usepackage{xcolor}


\makeatletter
\@ifpackageloaded{hyperref}{}{%
\ifxetex
  \usepackage[setpagesize=false, % page size defined by xetex
              unicode=false, % unicode breaks when used with xetex
              xetex]{hyperref}
\else
  \usepackage[unicode=true]{hyperref}
\fi
}
\@ifpackageloaded{color}{
    \PassOptionsToPackage{usenames,dvipsnames}{color}
}{%
    \usepackage[usenames,dvipsnames]{color}
}
\makeatother
\hypersetup{breaklinks=true,
            bookmarks=true,
            pdfauthor={ ()},
             pdfkeywords = {},  
            pdftitle={POSC 2103--007: Introduction to United States
Government},
            colorlinks=true,
            citecolor=blue,
            urlcolor=blue,
            linkcolor=magenta,
            pdfborder={0 0 0}}
\urlstyle{same}  % don't use monospace font for urls


\setcounter{secnumdepth}{0}





\usepackage{setspace}

\title{\LARGE\bfseries\sffamily\color{astate} POSC 2103--007:
Introduction to United States Government}
\author{Dr.~Cameron Wimpy}
\date{Spring 2020}


\begin{document}  
	% \vspace*{1\baselineskip}
	% \hspace*{-0.1\textwidth}{\centering\includegraphics[scale=.5]{astate.png}}
	
		\maketitle
		
	
		\thispagestyle{firststyle}

%	\thispagestyle{empty}


	\noindent \begin{tabular*}{\textwidth}{ @{\extracolsep{\fill}} lr @{\extracolsep{\fill}}}


\faEnvelopeO \hspace{.2em}\texttt{\href{mailto:cwimpy@astate.edu}{\nolinkurl{cwimpy@astate.edu}}}     & \faGlobe \hspace{.2em}\href{http://cwimpy.com}{\tt cwimpy.com}\\
\faHourglass \hspace{.2em}MW: 10:00--11:30 a.m. \textbar{} R:
9:30--11:30 a.m.  & \faClockO \hspace{.2em}MW 12:30--1:45 p.m.\\
\faHome \hspace{.2em}HSS:
3022                 & \faLocationArrow \hspace{.2em}HSS: 3005\\
\faUserPlus \hspace{.2em}TBD                     & \faSlack \hspace{.2em}TBD\\
	&  \\
	\hline
	\end{tabular*}
\vspace{2mm}
	


\hypertarget{course-description}{%
\section{Course Description}\label{course-description}}

This course serves as an introduction to: \textbf{The constitution,
government, and politics of the United States.} The main goal of this
course is to help you understand the structure and processes of the
American political system. This is accomplished by reading about and
analyzing:

\begin{itemize}
\tightlist
\item
  the key structures and institutions of government;
\item
  relevant political actors such as political parties, interest groups
  and the media;
\item
  the electoral process;
\item
  voters and voting behavior.
\end{itemize}

\hypertarget{general-education-goal}{%
\section{General Education Goal}\label{general-education-goal}}

\begin{itemize}
\tightlist
\item
  \textbf{Developing a strong foundation in the social sciences}:
  Students should be aware of the diverse systems developed by humans to
  manage and structure our relationships with one another. Students
  should prepare for the full range of public and private roles they are
  expected to fulfill as citizens, decision-makers and human beings in a
  democratic America and in a global society.
\end{itemize}

\hypertarget{student-learning-outcomes}{%
\section{Student Learning Outcomes}\label{student-learning-outcomes}}

\begin{enumerate}
\def\labelenumi{\arabic{enumi}.}
\item
  Explain the processes and effects of individual and group behavior
\item
  Analyze events in terms of the concepts and relational propositions
  generated by the social science tradition
\item
  Have a working knowledge of the government of the United States
\item
  Understand the challenges of governing in the United States
\item
  Explain how individuals and groups interact to produce their
  collective experience
\item
  Analyze empirical evidence to make reasoned arguments about government
  of the United States
\end{enumerate}

\clearpage

\hypertarget{course-requirements}{%
\section{Course Requirements}\label{course-requirements}}

\hypertarget{textbook}{%
\subsection{Textbook}\label{textbook}}

This textbook is required. You can search for it online, in the
bookstore, and view the publisher's website via this ISBN:
\href{http://global.oup.com/ushe/product/american-government-9780190299903}{978-0190299903}.
The book includes a student companion website that has additional
materials to help you study and further explore government in the United
States.

\hangindent=0.7cm

Gitelson, Alan, Robert Dudley, and Melvin J. Dubnick (2017).
\emph{American Government: Myths and Realities 2016 Election Editions}.
New York: Oxford University Press.

\hypertarget{technology}{%
\subsection{Technology}\label{technology}}

You will need to use your A-State email and have access to Blackboard.
If there is any issue with these items you should me know by the second
day of class.

\hypertarget{exams}{%
\subsection{Exams}\label{exams}}

There will be three multiple choice exams. The exams will take place via
Blackboard (make sure you can access Blackboard by the second day of
class). Each exam will consist of 50 questions taken from the lectures
and readings. The final exam is \underline{not} comprehensive.

\hypertarget{current-events}{%
\subsection{Current Events}\label{current-events}}

Given the topic of this course we will begin each class with a brief
(and sometimes not so brief) discussion of current events as they relate
to government in the United States. Please be prepared to participate in
this discussion by reading the news at your favorite source. If you do
not have a favorite source, then I suggest looking at a range of
outlets. I often read: \emph{The New York Times}, \emph{The Washington
Post}, \emph{The Economist}, \emph{Fox News}, \emph{CNN}, \emph{BBC
News}, \emph{Real Clear Politics}, \emph{Five Thirty Eight}, and
\emph{Politico} among others. Please bring in any articles or topics you
would like to discuss. There are also some great podcasts/radio options
including: \emph{NPR}, \emph{FiveThirtyEight}, and \emph{The Daily}
(NYT). Please refrain from looking up current events in real time during
class. This is disruptive, rude, and signals a lack of preparation at
best. Come prepared with topics and details---please do not just blurt
out headlines.

\hypertarget{course-policies}{%
\section{Course Policies}\label{course-policies}}

These course policies are subject to change, but assume they are in
effect unless you hear otherwise from Dr.~Wimpy.

\hypertarget{academic-freedom}{%
\subsection{Academic Freedom}\label{academic-freedom}}

We are living in a particularly polarized period of American politics,
and that means that some of us will often disagree with differing (but
typically reasonable) points of view. Disagreement and debate is a
healthy part of our political system so please be respectful of your
classmates as they share their perspectives. Please know you have
complete academic freedom to express your opinions and thoughts on U.S.
Government. Having said that, any type of hateful or insulting
discussion will not be tolerated.

\hypertarget{grading-policy}\\
Exam \#1 & 25\%\\
\cellcolor{gray!6}{Exam \#2} & \cellcolor{gray!6}{25\%}\\
Quizzes & 25\%\\
\bottomrule
\end{tabular}
\end{table}

The grading system will follow the
\href{https://www.astate.edu/college/graduate-school/academic-policies/}{A-State
convention}. Letter grades will be assigned as follows:

\begin{table}
\centering
\begin{tabular}{ll}
\toprule
Letter Grade & Range\\
\midrule
\cellcolor{gray!6}{A:} & \cellcolor{gray!6}{89.5–100}\\
B: & 79.5–89.4\\
\cellcolor{gray!6}{C:} & \cellcolor{gray!6}{69.5–79.4}\\
D: & 59.5–69.4\\
\cellcolor{gray!6}{F:} & \cellcolor{gray!6}{00.0-59.4}\\
\bottomrule
\end{tabular}
\end{table}

\hypertarget{e-mail-policy}{%
\subsection{E-mail Policy}\label{e-mail-policy}}

I am usually quick to respond to student e-mails during regular business
hours. However, student e-mails tend to do several things that try my
patience. The following points outline why I will not respond to certain
e-mails students send. \emph{Note that some of these may not apply
specifically to this course.}

\begin{enumerate}
\def\labelenumi{\arabic{enumi}.}
\tightlist
\item
  The student could answer the inquiry by reading the syllabus.
\item
  The student provides am unsolicited reason for missing class (other
  than exam days). I do not need to know the exact reason why you missed
  class. Students with excusable absences are responsible for giving me
  a note that documents the reason for a missed exam. Anything else is
  just extra information.
\item
  The student wants to know what topics were missed during a
  missed/skipped class. The answer is always ``you missed what was on
  the syllabus.''
\item
  The student is not professional in the e-mail. Please consider writing
  professional e-mails, it's a good habit that will pay off one day.
  That means you begin the e-mail with a greeting and that you end it
  with a closing. You can learn more from sites like this:
  \url{https://www.indeed.com/career-advice/career-development/how-to-write-a-professional-email}
\item
  The student wants to know something that is readily discernible from
  Blackboard (e.g., ``how many discussion posts do I have left?,''
  ``what is my current grade?'').
\item
  The student is requesting an extension on an assignment for which the
  syllabus already established the deadline. The answer is always
  ``no.''
\item
  The student is
  \href{https://www.math.uh.edu/~tomforde/GradeGrubbing.html}{``grade
  grubbing''} or asking to round up a grade outside of the rounding
  outlined in the grading policy above. The answer is always ``no.''
\item
  The student is asking for an extra credit opportunity, a request that
  amounts to more grading for the professor. The answer is ``no.''
\end{enumerate}

\hypertarget{make-up-exam-policy}{%
\subsection{Make-Up Exam Policy}\label{make-up-exam-policy}}

There are no make-ups for missed exams unless there is a valid excuse.
Don't bother asking.

\hypertarget{extra-creditgrade-change-policy}{%
\subsection{Extra Credit/Grade Change
Policy}\label{extra-creditgrade-change-policy}}

I do not give extra credit and grades will only be changed in the event
of a calculation error. Do not bother asking about anything else.

\hypertarget{attendance}{%
\subsection{Attendance}\label{attendance}}

Attendance is required and I will take roll each class. See the
\href{http://www.astate.edu/a/student-conduct/student-standards/handbook-home.dot}{student
handbook} for the University policies on attendance.

\hypertarget{laptops-phones-tablets}{%
\subsection{Laptops, Phones, \& Tablets}\label{laptops-phones-tablets}}

Laptops and tablets are allowed in the classroom insofar as they are
used to take notes and/or accessing the online text (should you opt for
the online edition). Phones are not allowed to be out during class time
under any circumstances. If you have an emergency, please leave the
class and take care of it. Otherwise it can wait. If I see abuse of the
laptop/tablet policy I reserve the right to disallow these during class
time unless there is a documented and approved exception.

\hypertarget{academic-dishonesty-policy}{%
\subsection{Academic Dishonesty
Policy}\label{academic-dishonesty-policy}}

Plagiarism (the act of taking and/or using the ideas, work, and/or
writings of another person as one's own) is a serious offense against
academic integrity which could result in failure for the paper or
assignment, failure for the course, and/or expulsion from Arkansas State
University.

Cheating is an act of dishonesty with the intention of obtaining and/or
using information in a fraudulent manner. For further information,
including specifics about what constitutes plagiarism or cheating, see
ASU's Academic Integrity Policy in the student
\href{https://www.astate.edu/a/student-conduct/student-standards/handbook-home.dot}{handbook}.

\hypertarget{disabilities-policy}{%
\subsection{Disabilities Policy}\label{disabilities-policy}}

Any student in this course who has a disability that may prevent him/her
from fully participating in this course should contact Disability
Services at (870) 972-3964 (or here:
\url{https://www.astate.edu/disability}) as soon as possible so we can
make the necessary accommodations to facilitate your educational
opportunity.

\hypertarget{copyright-fair-use-statement}{%
\subsection{Copyright \& Fair Use
Statement}\label{copyright-fair-use-statement}}

The materials used in this course are subject to copyright laws. For
those materials which neither I nor ASU owns the copyright, I have
either obtained a license for use or am using the materials in a manner
that I reasonably believe is in compliance with the Fair Use exception
to the copyright laws. The other materials used in this course are
copyrighted. By this, I mean all materials generated for this class,
which include but are not limited to syllabi, exams, lectures, quizzes,
assignments, and any other document I post in Blackboard. Because these
items are copyrighted, you do not have the right to copy and distribute
any course materials unless I, or the document publishers, expressly
grant permission.

\newpage

\hypertarget{tentative-class-schedule}{%
\section{Tentative Class Schedule}\label{tentative-class-schedule}}

Important: class readings are subject to change, contingent on
mitigating circumstances and the progress we make as a class. Students
are encouraged to check the Blackboard announcements section and any
e-mails from me for updates.

\hypertarget{week-01-0113-0117-course-introduction-topic-introduction-u.s.-government}{%
\subsection{Week 01, 01/13 -- 01/17: Course Introduction, Topic
Introduction (U.S.
Government)}\label{week-01-0113-0117-course-introduction-topic-introduction-u.s.-government}}

\hypertarget{read}{%
\subsubsection{Read:}\label{read}}

\begin{itemize}
\item
  The Course Syllabus
\item
  Chapter 1
\end{itemize}

\hypertarget{week-02-0120-0124-the-constitution}{%
\subsection{Week 02, 01/20 -- 01/24: The
Constitution}\label{week-02-0120-0124-the-constitution}}

\hypertarget{read-1}{%
\subsubsection{Read:}\label{read-1}}

\begin{itemize}
\tightlist
\item
  Chapter 2
\end{itemize}

\hypertarget{week-03-0127-0131-federalism}{%
\subsection{Week 03, 01/27 -- 01/31:
Federalism}\label{week-03-0127-0131-federalism}}

\hypertarget{read-2}{%
\subsubsection{Read:}\label{read-2}}

\begin{itemize}
\tightlist
\item
  Chapter 3
\end{itemize}

\hypertarget{week-04-0203-0207-civil-liberties}{%
\subsection{Week 04, 02/03 -- 02/07: Civil
Liberties}\label{week-04-0203-0207-civil-liberties}}

\hypertarget{read-3}{%
\subsubsection{Read:}\label{read-3}}

\begin{itemize}
\tightlist
\item
  Chapter 4
\end{itemize}

\hypertarget{week-05-0210-0214-civil-rights}{%
\subsection{Week 05, 02/10 -- 02/14: Civil
Rights}\label{week-05-0210-0214-civil-rights}}

\hypertarget{read-4}{%
\subsubsection{Read:}\label{read-4}}

\begin{itemize}
\tightlist
\item
  Chapter 5
\end{itemize}

\hypertarget{week-06-0217-0221-exam-1}{%
\subsection{Week 06, 02/17 -- 02/21: Exam
1}\label{week-06-0217-0221-exam-1}}

\hypertarget{read-5}{%
\subsubsection{Read:}\label{read-5}}

\begin{itemize}
\tightlist
\item
  Chapters 1--5 (as needed to review)
\end{itemize}

\hypertarget{activities}{%
\subsubsection{Activities:}\label{activities}}

\begin{itemize}
\item
  Exam 1 Review (M) \textbar{} Please be prepared with questions!
\item
  Exam 1 (W) \textbar{} Covers Chapters 1--5
\end{itemize}

\hypertarget{week-07-0224-0228-public-opinion-and-political-participation-political-parties}{%
\subsection{Week 07, 02/24 -- 02/28: Public Opinion and Political
Participation \textbar{} Political
Parties}\label{week-07-0224-0228-public-opinion-and-political-participation-political-parties}}

\hypertarget{read-6}{%
\subsubsection{Read:}\label{read-6}}

\begin{itemize}
\tightlist
\item
  Chapters 6--7
\end{itemize}

\hypertarget{week-08-0302-0306-campaigns-and-elections}{%
\subsection{Week 08, 03/02 -- 03/06: Campaigns and
Elections}\label{week-08-0302-0306-campaigns-and-elections}}

\hypertarget{read-7}{%
\subsubsection{Read:}\label{read-7}}

\begin{itemize}
\tightlist
\item
  Chapter 8
\end{itemize}

\hypertarget{week-09-0309-0313-interest-groups}{%
\subsection{Week 09, 03/09 -- 03/13: Interest
Groups}\label{week-09-0309-0313-interest-groups}}

\hypertarget{read-8}{%
\subsubsection{Read:}\label{read-8}}

\begin{itemize}
\tightlist
\item
  Chapter 9
\end{itemize}

\hypertarget{week-10-0316-0320-media-politics}{%
\subsection{Week 10, 03/16 -- 03/20: Media \&
Politics}\label{week-10-0316-0320-media-politics}}

\hypertarget{read-9}{%
\subsubsection{Read:}\label{read-9}}

\begin{itemize}
\tightlist
\item
  Chapter 10
\end{itemize}

\hypertarget{week-11-0323-0327-spring-break-no-class}{%
\subsection{Week 11, 03/23 -- 03/27: Spring Break \textbar{} No
Class!}\label{week-11-0323-0327-spring-break-no-class}}

\hypertarget{week-12-0330-0403-exam-2}{%
\subsection{Week 12, 03/30 -- 04/03: Exam
2}\label{week-12-0330-0403-exam-2}}

\hypertarget{read-10}{%
\subsubsection{Read:}\label{read-10}}

\begin{itemize}
\tightlist
\item
  Chapters 6--10 (as needed to review)
\end{itemize}

\hypertarget{activities-1}{%
\subsubsection{Activities:}\label{activities-1}}

\begin{itemize}
\item
  Exam 2 Review (M) \textbar{} Please be prepared with questions!
\item
  Exam 2 (W) \textbar{} Covers Chapters 6--10
\end{itemize}

\hypertarget{week-13-0406-0410-congress}{%
\subsection{Week 13, 04/06 -- 04/10:
Congress}\label{week-13-0406-0410-congress}}

\hypertarget{read-11}{%
\subsubsection{Read:}\label{read-11}}

\begin{itemize}
\tightlist
\item
  Chapters 11
\end{itemize}

\hypertarget{week-14-0413-0417-the-presidency}{%
\subsection{Week 14, 04/13 -- 04/17: The
Presidency}\label{week-14-0413-0417-the-presidency}}

\hypertarget{read-12}{%
\subsubsection{Read:}\label{read-12}}

\begin{itemize}
\tightlist
\item
  Chapter 12
\end{itemize}

\hypertarget{week-15-0420-0424-bureaucracy}{%
\subsection{Week 15, 04/20 -- 04/24:
Bureaucracy}\label{week-15-0420-0424-bureaucracy}}

\hypertarget{read-13}{%
\subsubsection{Read:}\label{read-13}}

\begin{itemize}
\tightlist
\item
  Chapter 13
\end{itemize}

\hypertarget{week-16-0427-0501-courts-judges-the-law}{%
\subsection{Week 16, 04/27 -- 05/01: Courts, Judges, \& The
Law}\label{week-16-0427-0501-courts-judges-the-law}}

\hypertarget{read-14}{%
\subsubsection{Read:}\label{read-14}}

\begin{itemize}
\tightlist
\item
  Chapter 14
\end{itemize}

\hypertarget{activities-2}{%
\subsubsection{Activities:}\label{activities-2}}

\begin{itemize}
\tightlist
\item
  Exam 3 Review (W) \textbar{} Please be prepared with questions!
\end{itemize}

\hypertarget{week-17-0504-0508-exam-3}{%
\subsection{Week 17, 05/04 -- 05/08: Exam
3}\label{week-17-0504-0508-exam-3}}

\hypertarget{read-15}{%
\subsubsection{Read:}\label{read-15}}

\begin{itemize}
\item
  Chapters 11--14 (as needed to review)
\item
  Exam 3 (M) \textbar{} Covers Chapters 11--14
\end{itemize}

\hypertarget{activities-3}{%
\subsubsection{Activities:}\label{activities-3}}

\begin{itemize}
\tightlist
\item
  \textbf{The Final Exam is Monday, May \nth{4} @12:30 p.m.}
\end{itemize}





\vfill{}
\vfill{}
\HRule
%%%%%%%%%%%%%%%%%%%%%%%%%%%%%%%%%%%%%%%%%%%%%%%%%%%%%%%%%%%%%%%%%%%%%%%%%%%%%%%%%%%%%%%%%%%%%%%%%%%%%%%%%%%%%%%%%%%%%%%%%%%%%%%%%%%%%%%%%%%
\begin{center}
	{\scriptsize  Last updated: \today\- •\- \href{https://cwimpy.com}{cwimpy.com}}
 \end{center}
\end{document}

\makeatletter
\def\@maketitle{%
  \newpage
%  \null
%  \vskip 2em%
%  \begin{center}%
  \let \footnote \thanks
    {\fontsize{18}{20}\selectfont\raggedright  \setlength{\parindent}{0pt} \@title \par}%
}
%\fi
\makeatother
